%%%%%%%%%%%%%%%%%%%%%%%%%%%%%%%%%%%%%%%%%%%%%%%%%%%%%%%%%%%%%%%%%%%%%%%%%%%%%%%%
%2345678901234567890123456789012345678901234567890123456789012345678901234567890
%        1         2         3         4         5         6         7         8

\documentclass[a4paper, 10pt, conference]{IEEEtran}  % Comment this line out if you need a4paper

%\documentclass[a4paper, 10pt, conference]{ieeeconf}      % Use this line for a4 paper

\IEEEoverridecommandlockouts                              % This command is only needed if
                                                          % you want to use the \thanks command


% See the \addtolength command later in the file to balance the column lengths
% on the last page of the document

% The following packages can be found on http:\\www.ctan.org
\usepackage{graphics} % for pdf, bitmapped graphics files
\usepackage{epsfig} % for postscript graphics files
\usepackage{mathptmx} % assumes new font selection scheme installed
\usepackage{times} % assumes new font selection scheme installed
\usepackage[utf8]{inputenc}
\usepackage[onelanguage,ruled,linesnumbered]{algorithm2e}
\usepackage{booktabs} % For formal tables
\usepackage{amsmath}
\usepackage{amssymb}
\usepackage{lipsum}
\usepackage{todonotes}
\usepackage{xcolor}
\usepackage{subfigure}
\usepackage{multirow}

\hyphenation{cen-tra-li-zed}

\newcommand{\packdrop}{$PackDrop$\xspace}
\newcommand{\batchsend}{$BatchSend$\xspace}
\newcommand{\batchassembly}{$BatchAssembly$\xspace}

\newcommand{\distributedlb}{$Distributed$\xspace}
\newcommand{\dummylb}{$Dummy$\xspace}
\newcommand{\greedylb}{$Greedy$\xspace}
\newcommand{\refinelb}{$Refine$\xspace}
\newcommand{\charm}{{\small\texttt{Charm++}}\xspace}

\newcommand\tofix[1]{\textcolor{orange}{#1}}

\title{A Batch Task Migration Approach\\ for Decentralized Global Rescheduling}

% Organizar as linhas de autores de forma diferente. Está muito caótico com múltiplas pessoas da mesma instituição as vezes separadas e as vezes juntas

\author{
\IEEEauthorblockN{Vinicius Freitas$^\star$, Alexandre de L. Santana$^\star$, Márcio Castro$^\star$, Laércio L. Pilla$^{\star\dagger}$\thanks{This work was partially supported by the  Brazilian Federal Agency for the Support and Evaluation of Graduate Education (CAPES) and by the Brazilian Council of Technological and Scientific Development (CNPq), project grant 401266/2016-8.}}
\IEEEauthorblockA{$^\star$ Universidade Federal de Santa Catarina (UFSC), Florianópolis, Brazil\\
$^\dagger$ Univ. Grenoble Alpes, Inria, CNRS, Grenoble INP, LIG, France\\
Email: {\smaller\texttt{\{vinicius.mctf,alexandre.santana\}@posgrad.ufsc.br, marcio.castro@ufsc.br, laercio.lima@inria.fr}}}
}

\begin{document}



\maketitle
\thispagestyle{empty}
\pagestyle{empty}


%%%%%%%%%%%%%%%%%%%%%%%%%%%%%%%%%%%%%%%%%%%%%%%%%%%%%%%%%%%%%%%%%%%%%%%%%%%%%%%%
\begin{abstract}

  Effectively mapping tasks of High Performance Computing (HPC) applications on parallel systems is crucial to assure substantial performance gains.
  As platforms and applications grow, load imbalance becomes a priority issue. 
  Even though centralized rescheduling has been a viable solution to mitigate this problem, its efficiency is not able to keep up with the increasing size of shared memory platforms.
  To efficiently solve load imbalance today, and in the years to come, we should prioritize decentralized strategies developed for large scale platforms.
  In this paper we propose our \textit{Batch Task Migration} approach to improve decentralized global rescheduling, ultimately reducing communication costs and preserving task locality.
  We implemented and evaluated our approach in two different parallel platforms, using both synthetic workloads and a molecular dynamics (MD) benchmark. 
  Our solution was able to achieve speedups of up to 3.75 and 1.15 on rescheduling time, when compared to other centralized and distributed approaches, respectively. 
  Moreover, it improved the execution time of MD by factors up to 1.34 and 1.22 when compared to a scenario without load balancing on two different platforms.

\end{abstract}

\begin{IEEEkeywords}
High Performance Computing, Global Rescheduling, Load Balancing, Performance Evaluation, Parallel Algorithms
\end{IEEEkeywords}

\section{Introduction}

%\begin{itemize}
%	\item Motivation: Why load balancing
%	\item Context: Scenarios in which global rescheduling applies
%	\item Problem: Scaling iterative applications: algorithm limitations
%	\item Justification: Scalable load balancing, distributed rescheduling
%	\item Related work: Hierarchical (limited scaling), Distributed (excessive comm), Diffusive (iterative), Refinement Based (centralized)
%	\item Proposed solution: Batch task migration to mitigate communication costs
%	\item Paper contributions	
%	\item Results brief
%	\item Paper structure
%\end{itemize}

%Growth of systems and applications.
%Need to achieve high scalability to use platforms.
%Efficient scheduling and use of resources.
%
%Dynamic iterative applications.
%Unable to use work stealing.
%Global rescheduling as a solution.
%
%Strong scaling of applications.
%Cost of centralized load balancing.
%Using platform parallelism to achive performance.
%Communication costs as a relevant overhead.
%
%Present hierarchical LB.
%Pros: weak scaling, topology.
%Cons: high comms, bottlenecks.
%Present distributed LB.
%Pros: parallel, refinement.
%Cons: high comms, network contention.
%
%\textbf{Cut out topo aware and diffusive, not related enough.}
%
%Brief intro to BTM.
%Reduced communication.
%Highly parallel.
%Fast convergence.
%
%Present contributions.
%
%Brief results.
%
%Present paper structure.
%
%\hline

Parallel machines are at their best when work is evenly distributed among compute nodes, and idle time is merely a myth.
Unfortunately, strong scaling applications for these platforms has been a challenge as long as they have existed.
In this context, uneven distribution of work and high communication overheads are the main villains when developing parallel applications~\cite{Deveci2015, commaware}.
Concerns towards these problems increase as systems grow in size and performance, consuming more resources, specially power, to solve some the most complex problems in scientific computing~\cite{exapower2015,padoin2017energy}.

Applications such as simulations of molecular dynamics (MD) and hydrodynamics suffer from load imbalance due to their intrinsic dynamic and iterative nature~\cite{namd,IPDPS13:LULESH}.
Although rescheduling algorithms have been able to greatly improve applications~\cite{namd0}, new approaches are needed to guarantee their performance as parallel systems grow.
Since mapping work to processing elements (PEs) is a NP-Hard problem~\cite{npcomplete}, the increase in application data and platform size makes centralized approaches to rescheduling inefficient.
This creates a need for scalable and decentralized rescheduling approaches, avoiding both excess of data to process and network contention~\cite{trahay2009scalable}.

The two main paths to achieve scalability in global rescheduling for iterative applications are Hierarchical and Distributed approaches.
Hierarchical load balancing explores parallelism using different approaches for fine-grain and coarse-grain steps~\cite{hybrid}.
These can scale, but are usually tied to the same limitations as the Centralized, as data is still aggregated in parent nodes.
On the other hand, Distributed load balancing seeks to achieve scalability in a decentralized fashion.
These scale better, but have limited system information and may incur in high amounts of communication.

Few are the Distributed strategies in the domain of global rescheduling, but their effectiveness is notable~\cite{grapevine,diffus}.
In this paper, we present the concept of \textit{Batch Task Migration} and a novel distributed global rescheduling algorithm that applies this technique, \textit{PackDrop}.
Our approach is based on the idea of grouping tasks prior to migration decisions in batches, decreasing communication overhead in algorithm decision time and enhancing locality of migrated tasks.

% Main contributions
In this paper, we present the following contributions: 
\begin{enumerate}
	\item A \textit{Batch Task Migration} approach for distributed rescheduling algorithms, which presents high scalability and task affinity potential.
	\item A novel distributed rescheduling algorithm, using our \textit{Batch Task Migration} approach, \textit{PackDrop}.
	\item An implementation of our algorithm, as well as a performance evaluation of this implementation.
\end{enumerate}

The remainder of this paper is divided as follows:
Section~\ref{sec:rw} presents recent work in dynamic rescheduling of scientific applications. 
Section~\ref{sec:algo} presents our novel approach and the developed algorithms. 
Section~\ref{sec:analysis} is a complexity analysis of \textit{PackDrop}, our distributed algorithm. 
Section~\ref{sec:impl} displays implementation details, execution environments and benchmarks used in this paper. 
Section~\ref{sec:eval} displays our performance evaluation and discussed experiments. 
And Section~\ref{sec:conclusion} presents the conclusion of this work and our plans for future research.



\section{Related Work} \label{sec:rw}

Global rescheduling is a well studied theme in high performance computing~\cite{Deveci2015,grapevine,nuco,hybrid,ZoltanParHypRepart07,diffus,Zheng2010}. 
In the centralized domain, strategies implement a variety of heuristics in order to achieve an homogeneous distribution of load.
Although centralized algorithms are used the most, they lack scalability and new approaches must be pursued as systems grow.

Refinement based solutions in centralized load balancing are very effective in dealing with low imbalance or high migration costs~\cite{MenonPHD}. 
Since these strategies use a limited migration heuristic, they are able to efficiently balance load with a low overhead.
However, since refine strategies limit a total number of migrations, they cannot deal with very unbalanced systems.
\todo[inline]{Citar balanceadores centralizados é mesmo necessário? Não parecem faltar referencias de distribuídos para colocar algo distante como relacionado}

Different approaches have been used to scale scheduling strategies.
Hierarchical algorithms will work differently in different levels, exploring parallelism and delivering better performance~\cite{nuco,hybrid,GengbinThesis}.
Strategies in this domain have been able to ensure scalability so far, but are still limited by a centralized beginning and data dependency.

Also in the hierarchical domain, multilevel graph partitioning has been used to efficiently reschedule tasks~\cite{ZoltanParHypRepart07}.
The hypergraph abstraction accurately represents communication, which provides more precision for load balance~\cite{commaware,Bathele2011graph}.
However, the cost of this representation is very high for a first task mapping process, being more recommended for repartitioning.

Work stealing is one of the most broadly used techniques for balancing load in parallel systems~\cite{DBLP:journals/ijpp/YangH18,Janjic2013}.
The completely distributed essence of this techniques make them very effective scheduling solutions for parallel and distributed irregular applications.

Diffusive load balancing is a completely distributed solution that may benefit iterative applications~\cite{diffus}.
These schedulers irradiate work from overloaded PEs to their neighbors, in an effort to achieve load balance in an iterative fashion.
Very unbalanced systems suffer with this kind of approach, since the scheduler may take too long to reach a solution.

\textit{Grapevine} is a completely distributed refinement-based scheduling solution.
It uses probabilistic transfer of load and epidemic communication protocols to achieve scalability~\cite{grapevine}.
We intend to use what was created in \textit{Grapevine} and reduce its communication, using the \textit{Task Packing} approach.

\textit{BinLPT} is a centralized workload-aware loop scheduler that adopts a greedy bin packing heuristic to adaptively partition the iteration space of irregular parallel loops into several chunks so as to amortize load imbalance while minimizing the number of chunks that are produced~\cite{Castro-Penna-WSCAD:2017}. Thus, runtime scheduling overheads can be reduced and iteration affinity may be exploited efficiently. Then, it assigns the heaviest chunks to the least overloaded cores, and then iteratively assigns lighter chunks to more heavily loaded ones. Finally, whenever a core finishes computing all its chunks, it steals a chunk from other one that still has work left.

\todo[inline]{Add related work to MigBSP~\cite{MigPFrighi} \@ Pilla}

\todo[inline]{Add a final paragraph to explain why our approach solve the aforementioned problems.}
\section{Batch Task Migration Approach} \label{sec:algo}

%Global rescheduling algorithms must be effective in order to improve scheduling, but should also have low overhead in order to avoid reducing its benefits.
%To ensure a quick and informed remapping of tasks, we present the \textit{Task Packing Approach}.
%Sharing global information (e.g. processor affinity, estimated computational load, expected communication patterns, etc) allows a PE to create groups of the best tasks to leave a given processor and take the migration decision with one message, instead of several.

The role of the global rescheduler is to ultimately reduce the application makespan. 
Thus, the scheduler policy must incur low overheads as to not overshadow it's benefits. 
Moreover, we envision that a \textit{Batch Task Migration} approach can ensure a quick and informed remapping of tasks, ultimately reducing the amount of messages during the scheduling decision process. 
Therefore, this section is dedicated to present our \textit{PackDrop} strategy as a distributed refinement-based technique that implements our proposed \textit{Batch Task Migration} approach.

%In this work we present the \textit{PackDrop} strategy, a distributed refinement-based technique implementing our new approach. 
The main benefits we expect with our approach are: 
\begin{enumerate}
	\item \textit{Reducing} unnecessary communication;
	\item \textit{Exploring locality} of tasks mapped to the same PE, since migrations are done in groups;
	\item An \textit{accelerated decision making process}, consequence of the first;
	\item And a diminished total application runtime.
\end{enumerate}

First we will explain the \textit{Batch Assembly} (Algorithm~\ref{alg::packcreation}) and the \textit{Batch Sending} (Algorithm~\ref{alg::packsend}) processes of this strategy.
Then the complete strategy will be presented in Algorithm~\ref{algo:packdrop}, $PackDrop$.

For simplicity, in all algorithms presented here: i) the symbol:~``$\rightarrow$'' will represent a remote procedure call; ii) and the symbol:~``$\Rightarrow$'' will  represent the start of a reduction process.

\subsection{Batch Assembly} \label{sec:algo:creation}

The \textit{Batch Assembly} process for the \textit{PackDrop} strategy is presented in Algorithm~\ref{alg::packcreation}.
It uses an estimated batch size ($s$), a set of local tasks ($T$), the current PE $load$ and a threshold for PE loads ($thrs$), to create a set of leaving tasks ($LT$).
The threshold is used to calculate an upper bound ($ub$) of the average system load ($\overline{load}$), using Equation~\ref{eq:ceil}. 
The load of any set of tasks is given by Equation~\ref{eq:load}.

\begin{equation}
	ceil(l,v) = (1+v)\times l
    \label{eq:ceil}
\end{equation}

\begin{equation}
	load_{set}(B) = \sum_{t \in B}t\ \ |\ \ B \text{ is a set of tasks}
	\label{eq:load}
\end{equation}

With this information, each PE will take the task with the lower load within its pool, and pack it (lines~$3-6$).
Then, if the sum of all tasks in the pack is greater then the expected batch size ($s$), the batch is assembled and the strategy starts creating another one (lines~$7-10$).
The process is repeated while $ub$ is greater than $load$~(line~$2$).

\begin{algorithm}[!ht]
    \DontPrintSemicolon
    \KwIn{$s$; $  T$; $load$; $thrs$}
    \KwOut{$LT$}
    $L \gets \varnothing,\ LT \gets \varnothing,\ ub \gets ceil(\overline{load},thrs)$ \\
    \While{$load > ub$}{
        $t \gets a \in   T\ |\ a$ is the lower bound of $T$\\
        $  T \gets   T\ \backslash\ \{t\} $\\
        $\textit{L} \gets L\ \cup\ \{t\}$\\
        $load \gets load - t$\\
        \If{$load_{set}(L) > s$}{
            $LT \gets LT\ \cup\ L$\\
            $L \gets \varnothing $
        }
    }
    $LT \gets LT\ \cup\ L$   
    \caption{Batch Assembly} 
    \label{alg::packcreation}
\end{algorithm}

Any unfinished $LT$s should be sent even if those are not complete.
This is done to avoid having an overloaded PE that can still migrate tasks.
A PE that receives this load will not receive as much load as others, but since the PE will not overload, it should not be prejudicial.

\subsection{Batch Sending} \label{sec:algo:sending}

The \textit{Batch Sending} process is presented in Algorithm~\ref{alg::packsend}.
The algorithm will use the $LT$s, produced by \textit{Batch Assembly}, and the set of $Targets$, produced by an information propagation step ($Gossip$~\cite{gossip}), in order to schedule packs on remote PEs.
This will produce a set of expected Batch/Target ($BG$) pairs, which should be confirmed by the remote target.

While the local PE still has available $LT$s to send (line~$2$), it will choose a random target ($G$) for it (line~$4$) and invoke a remote $Send$ procedure on its target ($g$) (line~$5$).
The selected set of leaving tasks ($b$) will be removed from the $LT$ set (line~$6$) and paired up with its target on the $BG$ set~(line~$7$), waiting for confirmation.
This process is repeated until all elements in $LT$ have attempted a $Send$.

\begin{algorithm}[!ht]
    \DontPrintSemicolon
    \KwIn{$LT$, $G$}
    \KwOut{$BG$}
    $BG \gets \varnothing$ \\
    \While{$LT \neq \varnothing$}{
        $b\ \gets p\ |\ p\ \in\ LT$\\
        $g\ \gets  rand(G)$\\
        $Send(b)\rightarrow g$\\
        $LT\ \gets\ LT\ \backslash\ \{b\}$\\
        $\textit{BG} \gets BG\ \cup\ \{(b,\ g)\}$\\
    }  
    \caption{Batch Sending}  
    \label{alg::packsend}
\end{algorithm}

When $Sends$ receive a negative response, Algorithm~\ref{alg::packsend} can be refed with the failed attempts and initiate another round of sends, until every member of $LT$ is migrated.

\subsection{PackDrop Algorithm} \label{sec:algo:main}

The PackDrop strategy is presented in Algorithm~\ref{algo:packdrop}.
For the sake of simplicity, in the explanation of this algorithm \textit{packs} will be a short for ``set of leaving tasks" (seen in Sections~\ref{sec:algo:creation}~and~\ref{sec:algo:sending}).

It will run individually on each PE, in a distributed fashion. 
Using a current local mapping of tasks to PEs ($M$), local load ($load_{local}$), a local PE identification ($MyId$) and knowing all PEs within the system ($P$), to produce a new mapping ($M'$).
The mapping of tasks is defined by Equation~\ref{eq:taskmap} as a set of pairs ($task, PE$), describing the location of tasks.
A local map of tasks contains only tasks that are assigned to the current PE.

\begin{equation}
	M:\ T \rightarrow P
	\label{eq:taskmap}
\end{equation}

The first part of the algorithm (lines $1-6$) is the information sharing and setup process. 
This process is done through $2$ global reductions of average PE load (line~$2$) and global number of tasks (line~$3$).
In this implementation we used two constants: $0.05$, in order to limit the imbalance at $5\%$ (on line~$5$), and $2$, in order to regulate the size of packs (on line~$6$).

After setup PEs are divided between two different workflows (line~$7$).
At this time, \textit{overloaded} PEs will start the Batch Creation process (line~$8$), further explained in Algorithm~\ref{alg::packcreation}.
Meanwhile, \textit{underloaded} PEs will initiate a \textit{Gossip Protocol}~\cite{gossip} in order to inform other elements they are willing to receive work (line~$11$).
\textit{Gossip} is a well known epidemic algorithm used to spread information on a system, providing fast convergence and almost-global awareness of what was shared.

After the information propagation, each PE must synchronize to start the remap (line~$13$). 
At this point, the remapping process will begin.
PEs will send their packs using Algorithm~\ref{alg::packsend}, \textit{Batch Send}, asynchronously (line~$14$).
After a pack is sent, a PE will accept or reject it based on their current load, this is done via \textit{three-way handshake}, so both parts confirm the migration.

If one or more packs were not successfully exchanged, an \textit{overloaded} PE must attempt a new \textit{Batch Send}, in order to achieve load balance, as specified in Algorithm~\ref{alg::packsend}.
Once the PEs know their new mappings, tasks are migrated and the strategy is finished (line~$15$). 

\begin{algorithm}
	\DontPrintSemicolon
    \KwIn{$M$, $load_{local}$, $P$, $MyId$}
    \KwOut{$  M'$}
    $  M' \gets \varnothing$\\
    $al \gets (AveragePeLoadReduction(load_{local})\Rightarrow  P)$ \\
    $tc \gets (TotalTaskCountReduction(| M|)\Rightarrow  P)$\\
    //Average~task~size~\quad//5\%~precision~on~balance\qquad
    $ats\gets \frac{al}{tc}$ \qquad\qquad\  $thrs \gets 0.05$\\
    $ub \gets ceil(al,thrs)$ \qquad //Upper migration threshold\\
    $s \gets ats\times (2-\frac{|  P|}{tc}$) \qquad\qquad //Pack load\\
    \uIf{$l > ub$}{
    	$packs \gets BatchCreation(ps,  T(  M),l,thrs)$
    }
    \Else{
    	$packs \gets \varnothing$\\
    	$T \gets (Gossip \rightarrow  P)$ \qquad //Targets for migration\\
    }
    $---Synchronization Barrier---$\\
    //Requests are processed as they are received back
    $R \gets BatchSend(packs, T)$\\
    $TaskMap(R,   M, MyId)$
    \caption{PackDrop}
    \label{algo:packdrop}    
\end{algorithm}

\textit{PackDrop} intends to remap tasks to PEs in a distributed, workload-aware fashion.
This approach is the basis for new batch task migration distributed strategies that may take other factors into account.


\section{Analysis of the Algorithm}

This section presents an analysis of the Parallel Algorithm~\ref{algo:packdrop}~(PackDrop). 
The complexity of the information propagation \textit{Gossip} have being evaluated as $O(log_fn)$ on previous work~\cite{grapevine},
where $f$ is the fanout for the algorithm and $n$ is the number of PEs in the system.

For the sake of simplicity, in the remainder of this analysis, the number of tasks in the system
will be refered to as $m$, and the costs for computation and communication will be represented as $p_c$ and $c_c$, respectively.
We also assume $c_c > p_c$ for all applying scenarios.
Unmentioned lines are assumed to have nonvarying cost, and thus will not interfer in the asymptotic analysis.

Lines~2~and~3 are reductions, which have a well known cost of $O(lg\ n)$ (equal to $O(log_2n)$).
Lines 8 and 11 are concurrent, so their cost will be the max among both:
\begin{equation*}
  max(T(PackCreation),T(Gossip))
\end{equation*}
We also know that the worst case for $PackCreation$ (Algorithm~\ref{alg::packcreation}) is rather unrealistic, 
since it would assume that a single PE contains $m$ tasks and a single task can have $load(task) > \overline{load}$, being $O(m-1)$, assuming $1$ would not be migrated, asymptotically, $O(m)$.

%Aside from this case, we could assume, for fixed $m$ applications and locality based overdecomposition, a complexity $O(m/n -t)$, 
%where $t$ are the tasks that stay on the PE so it remains balanced.
%The value of $t$ will vary with the imbalance of the entire system, the worst case would be $t=1$, so we assume $T(Pack Creation)\in O(m/n)$.

This takes lines 8 and 11 cost to:
\begin{equation*}
 max(O(p_cm),O(c_clog_fn))
\end{equation*}
and since $c_c$ is several orders of magnitude bigger than $p_c$, we could assume $T(PackCreation)\in T(Gossip)$, which makes the 
complexity of these lines to $O(log_fn)$.

Finally, line 14 will have its cost associated directly with the cost of Algorithm~\ref{alg::packcreation}, since it will account $c_c$ for every pack that was created then.
Let $ps$ be the mean number of tasks inside of a pack, and $m_l$ the maximum number of tasks in a given overloaded PE.
%This is where our approach presents gains in lessening communication costs.
At this step, a solution without Task Packing would have a $c_c$ associated with $O(m_l)$, of value $O(log_nm)$ in the average case.
This happens because the higher the $m/n$ ratio, the more tasks will be migrated in a decision process similar to the one made on Algorithm~\ref{alg::packcreation}, but as overdecomposition increases and imbalance decreases, $m_l$ is unable to grow as much, following a logarithmic curve. 

In our proposed solution, the complexity of line~14 will be divided by $ps$, enabling even higher scalability. Having a fanout $f$ of $2$, we have an asymptotic worst case of:
\begin{equation}
 T(PackDrop) = O(m/ps) + O(lg\ n)
 \label{eq:worstcase}
\end{equation}
and an average case of:
\begin{equation}
 T(PackDrop) = O(log_n(m/ps))+O(lg\ n) = O(lg\ n)
\end{equation}
\section{Implementation}

\textit{PackDrop} was implemented as a load balancing strategy in \texttt{Charm++}\footnote{Available at: \texttt{https://github.com/OMMITED-FOR-BLIND-REVIEW}}.
\texttt{Charm++} is a parallel programming model that provides a \textit{load balancing framework} based on migration of its parallel, message-driven objects, the \textit{chares}~\cite{CharmLOTR}.
\textit{Chares} are mapped as \textit{tasks} onto PEs and the \texttt{Charm++} runtime system (RTS) provides the load information needed for the \textit{PackDrop} strategy.

The \texttt{Charm++} RTS allows for the desired asynchronous behavior of \textit{PackDrop}.
It also provides necessary reduction and quiescence detection mechanisms, used in this implementation.
Reductions are used to evaluate the total number of chares and the average load in the system, while the quiescence detection is necessary to finalize the information propagation step of the algorithm.

\texttt{Charm++} provides application-independent load balancing, which means that any application (that implements a PUP framework~\cite{sc14charm}), may use global rescheduling strategies implemented for this RTS.
This means that any of the available applications for \texttt{Charm++} can be used to evaluate our strategy and compare it to other load balancers available for this RTS.


\subsection{Benchmarks} \label{sec:benchmarks}

We experimented our strategy with $2$ benchmarks that are publicly available for \texttt{Charm++}.
The first is a synthetic benchmark called \textbf{LB Test}. 
It simulates work with a variety of communication topologies, such as ring, meshes and randomized patterns.
\textit{LB Test} is known to have a low migration cost, with light \textit{tasks}, and most of its load bound to computation, instead of communication.

The second is a molecular dynamics (MD) mini-app called \textit{LeanMD}\footnote{Available at: \texttt{http://charmplusplus.org/miniApps}}.
This mini-app simulates the behavior of atoms and it mimics the force computation done in the state-or-the-art MD application NAMD, winner of the Golden Bell Award~\cite{grapevine}.
LeanMD uses geometric decomposition in a three-dimensional ($3$D) simulation space.
However, since the number of simulated atoms in each region affects the number of exchanged messages, it has an irregular and dynamic communication pattern, even though it respects the geometric distribution.

\subsection{Other schedulers}

\textit{PackDrop} was compared strategies available on the \texttt{Charm++} benchmark suite.
More specifically, strategies that may be selected by \texttt{Charm++}'s workload-aware \textit{Meta-balancer}~\cite{MenonPHD}.
Their behavior is briefly presented ahead.

\begin{itemize}
	\item \textit{Refine} is a refinement-based strategy that tries to minimize the number of migrated \textit{tasks}, exchanging load among PEs.
This strategy is specially efficient if the imbalance isn't too high.
	\item \textit{Greedy} creates two heaps, one for \textit{tasks} (max-min) and one for PEs (min-max). 
Then, it assigns tasks to PEs, putting the most work-heavy tasks on the least loaded PEs.
This strategy provides a good load balance, but may incur in high migration overhead.
    \item \textit{Distributed}, also known as \textit{Grapevine}, is a distributed strategy based on epidemic communication and probabilistic transfer of work.
   This strategy has good scalability, but does not performs as well as centralized ones in smaller scenarios.

\end{itemize}




\section{Performance Evaluation} \label{sec:eval}

Two platforms were for performance evaluation of our novel \textit{PackDrop} scheduler:
i) A tightly coupled \textit{Supercomputer} with $24$ (in $2$ separate NUMA nodes) PEs per node, using an Infiniband interconnection supported by Intel's Parallel Studio XE implementation of \texttt{MPI} (v2017.4).
ii) A smaller \textit{Cluster} with $4$ PEs per node, using a Gigabit Ethernet interconnection.
Details of both platforms are available on Table~\ref{tab:ptinfo}.

\begin{table}[ht]
    \centering
    \caption{Platform Information of each Node from Supercomputer and Cluster evaluations.}
	\begin{tabular}{l|l|l}
	Node Info.	 		& Supercomputer 		& Cluster \\ \hline
        CPUs	   			& $2\times12$ 			& $4$ \\
        Intel Xeon			& E5-2695v2 			& X3440\\
        CPU Freq.  			& $2.4$GHz   			& $2.53$GHz\\
        RAM        			& $64$GB			& $16$GB\\
        Network 			& Infiniband FDR 		& Gigabit Ethernet\\
        OS      			& RedHat Linux 6.4 		& Ubuntu 14.04\\
        \texttt{GCC}			& $5.3.1$			& $5.4.0$\\
        \texttt{Charm++} 		& $6.8.1$ 			& $6.8.1$\\
        \texttt{MPI}			& $3.1.0$			& -\\
        \texttt{GCC} Flags		& \texttt{-std=c++11 -O3} 	& \texttt{-std=c++11 -O3} \\
	\end{tabular}
    \label{tab:ptinfo}
\end{table}

Ahead we present the metrics used to compare our new global rescheduling strategy with \textit{Greedy}, \textit{Refine} and \textit{Distributed}.   
Then, we discuss results obtained in both platforms descripted in Table~\ref{tab:ptinfo} and the scalability of our proposed solution.
All raw data of our results, as well as parsing scripts for analysis are publicly available\footnote{Available at: \texttt{https://github.com/eclufsc/\\packdrop-data-analysis}.}.

\subsubsection*{Metrics}

\textit{Application time} is one of the most relevant metrics to evaluate load balancers in \texttt{Charm++}.
Since migrations may induce high overhead and impact communication costs, a bad algorithm may finish fast, but increase imbalance, and thus, application time.

\textit{Load balancer decision time}, although not the most relevant for the application itself, the decision time is an indicator of its scalability.
Some centralized schedulers, such as \textit{Greedy}, work very well on local machines, with a reasonable data input.
However, when executing on distributed memory environments, the scalability of centralized strategies is limited.

\subsection{Evaluation on Cluster} \label{sec:cluster}

All experiments executed on cluster were compiled with \texttt{Charm++} using the \texttt{--with-production} option, combined with the specifications detailed on Table~\ref{tab:ptinfo}.
$32$ homogeneous compute nodes were used, with a total of $128$ PEs.
In addition to previously mentioned schedulers, \textit{Dummy} was added as the representative of a situation with no remap of tasks, it only aggregates the information a centralized strategy needs (its cost is the base for every centralized strategy).

\subsubsection{Evaluation with Synthetic Load} \label{sec:cluster:lbtest}

\textit{LB Test} experiments had a total of $18990$ \textit{tasks}, executed over $150$ iterations, performing load balance every $40$ iterations.
\textit{Task} loads vary from $30$ms to $9000$ms, which provides reasonable imbalance of load, causing global rescheduling to be useful in this case.
Ring, 2D mesh and 3D mesh communication topologies were used to provide different levels of migration impact and communication cost.

\todo[inline]{Acho que não faz sentido apresentarmos tantas casas depois da vírgula. Não é como se tivéssemos um precisão nos tempos onde isso fizesse diferença.}

\begin{table}[t]
	\centering
    \caption{Average LB Test application time on the cluster execution.}
	\begin{tabular}{l | c  c  c}
    	Scheduler & Time (Ring) & Time (2D) & Time (3D) \\ \hline
        Distributed & $47.49298$s & $48.64839$s & $49.05481$s \\
        Greedy & $46.54101$s & $49.56052$s & $51.06850$s \\
        Dummy & $52.43016$s & $53.17254$s & $53.94068$s \\
        PackDrop & $46.81598$s & $47.37120$s & $47.97426$s \\
        Refine & $45.49095$s & $46.29277$s & $47.21924$s \\		
	\end{tabular}
    \label{tab:lbtest:apptime}
\end{table}

\todo[inline]{Figuras estão muito longas.}

%\begin{figure}
%	\begin{subfigure}[a]{0.45\linewidth}
%		\centering
%		\includegraphics[width=\linewidth]{images/apptime_lbtest_g5k.pdf}
%    		\caption{LB Test cluster execution results.}
%    		\label{fig:eval:g5k:lbtest:apptime}
%	\end{subfigure}
%	\begin{subfigure}[b]{0.45\linewidth}
%		\centering
%		\includegraphics[width=\textwidth]{images/apptime_lbtest_g5k.pdf}
%    		\caption{LB Test cluster execution results.}
%    		\label{fig:eval:g5k:lbtest:apptime1}
%	\end{subfigure}	
%	\caption{LB Test cluster execution results.}
%    	\label{fig:eval:g5k:lbtest:apptime2}
%\end{figure}

\begin{figure}
	\centering
	\subfigure[LB Test]{\includegraphics[width=0.45\linewidth]{images/apptime_lbtest_g5k.pdf}\label{fig:eval:g5k:lbtest:apptime}}	
	\subfigure[LeanMD]{\includegraphics[width=0.45\linewidth]{images/apptime_leanmd_g5k.pdf}\label{fig:eval:g5k:leanmd:time}}
	\caption{Cluster Execution Results for both applications.}
	\label{fofinho3}
\end{figure}

\todo[inline]{Análise estatística para poder dizer que esses tempos são verdadeiramente diferentes?}

Each configuration of the benchmark was executed $15$ times, with results depicted on Figure~\ref{fig:eval:g5k:lbtest:apptime} and Table~\ref{tab:lbtest:apptime}.
Results for \textit{Greedy} show how different communication topologies affect the scheduling performance.
Since \textit{Greedy} migrates many \textit{tasks}, the more they communicate, the more migrations impact the application time.

The increased in communication cost can be verified among all scheduling strategies, but in none as much as in \textit{Greedy}.
Our novel approach, \textit{PackDrop} has outperformed the other descentralized strategy, \textit{Distributed}, in the \textit{LB Test} case in this scale.
However, since the platform is not large enough to present all of the potential gains of decentralized strategies, \textit{Refine}, with a reduced migration count approach, still outperforms any other scheduler in this benchmark.
Nevertheless, this indicates a good scalability pontetial, specially in a cluster with high communication overhead, due to its Gigabit Ethernet connection.

\subsubsection{Evaluation with Molecular Dynamics} \label{sec:cluster:md}

\textit{LeanMD} experiments generated a $9\times9\times9$ space, with a total of $27702$ \textit{tasks}.
Each execution ran $500$ iterations, with a first rescheduling step at the $10$th iteration. 
Rescheduling periods (RP) of every $30$ (short) and every $60$ (long) iterations were used, providing different impacts of rescheduling on the application.
\textit{Greedy} and \textit{Dummy} were excluded from this evaluation due to their high cost in an application such as \textit{LeanMD}. 

\begin{table}[!ht]
	\centering
	\caption{LeanMD mean application time on the cluster execution.}	
	\begin{tabular}{l|c c c}
	Scheduler & Time (Short RP) & Time (Long RP) & Mean LB Time \\ \hline
	Distributed & $69.35606$s & $68.36055$s & $167.0444$ms \\ 
	PackDrop & $55.98428$s & $55.51554$s & $143.1028$ms \\ 
	Refine & $59.35696$s & $55.89895$s & $539.8364$ms \\ 
	\end{tabular}
	\label{tab:eval:g5k:leanmd:time} 
\end{table}

%\begin{figure}[!ht]
%	\centering
%    \includegraphics[width=0.43\textwidth]{images/apptime_leanmd_g5k.pdf}
%	\caption{LeanMD cluster execution results.}
%    \label{fig:eval:g5k:leanmd:time}
%\end{figure}

%\todo[inline]{Não acho que precisamos da figura dos tempos de escalonamento.}

%\begin{figure}[!t]
%	\centering
%    \includegraphics[width=0.43\textwidth]{images/schedtime_leanmd_g5k.pdf}
%	\caption{LeanMD cluster execution results.}
%    \label{fig:eval:g5k:leanmd:schedtime}
%\end{figure}

Each configuration of LeanMD was executed $10$ times, making a total of $5000$ steps per configuration and are depicted in Table~\ref{tab:eval:g5k:leanmd:time} and in Figure~\ref{fig:eval:g5k:leanmd:time}.
Observed application times presented a standard deviation from the mean lower than $2\%$ for all results presented.

Results show a better overall performance of \textit{PackDrop}, outperforming both compared strategies in the two scenarios chosen.
Since our strategy migrates groups of tasks, it improves locality of tasks after migration, outperforming \textit{Distributed} due to this.

The \textit{Mean LB Time}, presented in Table~\ref{tab:eval:g5k:leanmd:time}, shows the time taken by the periodical rescheduling (LB), task migration and the first iteration after the LB call.
It shows the increased cost of \textit{Refine}, which is due to both information agregation costs and dealing with the high amounts of application data in a centralized fashion.
\textit{PackDrop} displays its effectiviness in rescheduling time, outperforming both compared strategies, and resulting in an overall better application time. 

\subsection{Evaluation on Supercomputer} \label{sec:sdumont}

All experiments executed on supercomputer were compiled with \texttt{Charm++} using the \texttt{--with-production} option, combined with the specifications detailed on Table~\ref{tab:ptinfo}.
Different numbers of homogeneous $2\times 12$ PEs compute nodes ($2$ NUMA-nodes with $12$ cores each) were used to evaluate our strategy's scalability.
We ranged from $16$ ($384$ PEs) to $32$ ($768$ PEs) unique nodes in our evaluation. 

\subsubsection{Evaluation with Molecular Dynamics} \label{sec:sdumont:md}

\textbf{LeanMD} experiments generated a $10\times15\times10$ space, with a total of $171$K \textit{tasks}.
Each execution ran $100$ iterations, with a first rescheduling step at the $9$th iteration. 
Rescheduling was performed every $30$ iterations and each configuration of LeanMD was executed $10$ times, making a total of $1000$ steps per configuration. 

%\begin{table}
%	\centering
%	\caption{LeanMD Supercomputer Results.}
%	\begin{table}{l|c|c}
%	Scheduler 	& Application Time 	& LB Time \\ \hline
%	Dummy 		& $59.35696$s 		& $539.8364$ms \\ 
%	Greedy 		& 					& \\
%	Distributed	& $69.35606$s  		& $167.0444$ms \\ 
%	PackDrop 	& $55.98428$s 		& $143.1028$ms \\ 
%	
%	\end{table}
%	\label{tab:eval:sdumont:leanmd}
%\end{table}

\begin{figure}[!ht]
 \centering
 \subfigure[Application Time (s)]{\includegraphics[width=0.45\linewidth]{images/apptime_leanmd_sdumont.pdf}\label{fig:eval:sdumont:leanmd:apptime}}
 \subfigure[Rescheduling Time (s)]{\includegraphics[width=0.45\linewidth]{images/schedtime_leanmd_sdumont.pdf}\label{fig:eval:sdumont:leanmd:schedtime}}
 \caption{LeanMD supercomputer execution results.}
 \label{fig:eval:sdumont:leanmd}
\end{figure}

%\begin{figure}[!ht]
% \centering
% \includegraphics[width=0.9\linewidth]{images/apptime_leanmd_sdumont_bars.pdf}
% \caption{LeanMD supercomputer AppTime execution results.}
% \label{fig:eval:sdumont:leanmd:apptime:bars}
%\end{figure}

%\begin{figure}
%	\centering
%	\includegraphics[width=0.9\linewidth]{images/schedtime_leanmd_sdumont.pdf}
%	\caption{LeanMD supercomputer SchedTime execution results.}
%	\label{fig:eval:sdumont:leanmd:schedtime}
%\end{figure}

Results of mean application time are displayed in Figure~\ref{fig:eval:sdumont:leanmd:apptime} and mean rescheduling time in Figure~\ref{fig:eval:sdumont:leanmd:schedtime}.
\textit{Refine} was excluded from this evaluation since this LeanMD input presents more data than Refine is able to process in a \tofix{reasonable time}, being in average $1000$ms slower than \textit{Greedy} in the $384$ PEs test case. % Nenhuma ideia de tempo geral? O que é razoável?

\textit{Distributed} benefits from this platform due to the Infiniband low latency communication, which reflects on improved total application times, as seen in Figure~\ref{fig:eval:sdumont:leanmd:apptime}.
Our novel approach, \textit{PackDrop}, followed it closely and we can see that its rescheduling time in larger systems outperforms \textit{Distributed}, displayed in Figure~\ref{fig:eval:sdumont:leanmd:schedtime}.

The rescheduling and application time results of LeanMD in this platform highlight the importance of using scalable approaches to load balancing, as well as using available parallism in execution environments.
This is specially visible in \textit{Greedy} results on Figure~\ref{fig:eval:g5k:lbtest:apptime}, where the application performance was decreased after the global rescheduling process.
Increased migration costs and higher \textit{hop} counts in communication, consequences of load balancing, heavily impacted LeanMD in this case.

\subsection{Performance Evaluation Overview} \label{eval:overview}

Most scientific applications today seek strong scaling, increasing their computational platforms to solve problems faster.
Our results show that, to achieve such an objective, an application must implement efficient load balancing strategies.
We present \textit{PackDrop} as a solution for scalable rescheduling of work in distributed memory systems.

Section~\ref{sec:cluster:lbtest} shows the efficiency of our strategy. 
Results highlight the importance of load balancing even in synthetic loads.
The \textit{LB Test} benchmark used has very low migration and communication overhead, and most of its work is done locally, which is optimal for rescheduling evaluation of raw computational workload.
Our strategy was only outperformed by \tofix{\textit{Refine} in all test cases, which is expected, since the total load dealt with is considerably small ($\sim 19$K tasks) and migrations cheap, which benefits centralized schedulers.}

Sections~\ref{sec:cluster:md}~and~\ref{sec:sdumont:md} display evaluation of a MD mini-app, LeanMD (better described in Section~\ref{sec:benchmarks}).
This represents ``a real world-like" scenario, in which applications may have dynamic communication patterns and high migration overhead.
Results presented here highlight the overhead of centralized rescheduling approaches when joined with large-scale applications ($171$K tasks) and big environments, which increases work and information aggregation costs, respectively.

\textit{Distributed} outperformed our approach in the Supercomputer platform, due to its more refined approach for load balancing and high-speed network interconnection.
However, the results show that \textit{PackDrop} and its locality friendly batching of tasks for migration guarantees better performance in the Cluster platform, which portrays a Gigabit Ethernet interconnection.
\textit{PackDrop} was able to efficiently scale applications among all observed platforms, and had a faster rescheduling time than \textit{Distributed} in most of the observed cases.

\section{Conclusion} \label{sec:conclusion}

In this paper we have presented the \textit{Batch Task Migration} approach for distributed global rescheduling.
It intends to preserve task communication locality, migrating multiple work units from a source to the same destination, in order to balance system load.
This preserves communication efficiency, while other workload-aware strategies perform rescheduling without considering task locality.

Our approach also mitigates communication costs during algorithm execution time.
We guarantee this by transmitting information about multiple migrations at a time, in \textit{Batches}.
Thanks to this, our novel scheduler (presented in Section~\ref{sec:algo:main}) has an increased performance in high communication overhead platforms, discussed in Section~\ref{sec:cluster}.

We have evaluated our strategy in two different execution environments. 
The first was a high communication cost, $4$ cores/node cluster, executing over $32$ cores.
In this scenario, \textit{PackDrop} had a rescheduling speedup of up to $3.75$ and $1.15$ when compared to centralized and distributed approaches, respectively (Section~\ref{sec:cluster}).

The second scenario was a highly coupled cluster with low communication overhead, with $24$ cores/node.
We executed our experiments varying platform size from $16$ to $32$ nodes.
In this scenario, rescheduling time of \textit{PackDrop} and \textit{Distributed} were very similar, although both had a time up to $3$ orders of magnitude faster than any centralized approach. %Add some data to reinforce this
This reinforces the relevance of work in the distributed scheduling domain, and approaches such as our \textit{Batch Task Migration}.

\subsection{Future Work}

Future work on this theme includes the use of \textit{Batch Task Migration} in the communication-aware domain.
Since our approach already has locality-based benefits, combining this with communication pattern information may incur on even greater performance increase in applications~\cite{commaware}.
We believe a novel strategy focused on the \textit{Stencil} programming model is something to be considered, prioritizing migration of edges among PEs, instead of random parts of the stencil~\cite{stenciltiling}.

Further work will also be developed in order to increase performance in heterogeneous clusters.
These may have heterogeneous processing capacities and network capabilities, which enhances complexity of load balancing significantly~\cite{Beri2015hetws,Cheriere2015hetdist}.
In this given scenario, enhancing rescheduling decision processes may be crucial to ensure gains in application performance.

\section*{ACKNOWLEDGEMENTS}
%\todo[inline]{@Laércio, adicionar agradecimentos no contexto do projeto de pesquisa ao CNPq? -- Acknowledgements são coisas para nos preocuparmos com o artigo aceito. Não vejo como algo necessário para o momento. }
%\todo[inline]{Adicionar agradecimentos para ambas as bolsas de mestrado}

The authors acknowledge the National Laboratory for Scientific Computing (LNCC/MCTI, Brazil) for providing HPC resources of the SDumont supercomputer, which have contributed to the research results reported within this paper (see \texttt{http://sdumont.lncc.br}).

Experiments presented in this paper were carried out using the Grid'5000 testbed, supported by a scientific interest group hosted by Inria and including CNRS, RENATER and several Universities as well as other organizations (see \texttt{https://www.grid5000.fr}).

\bibliographystyle{IEEEtran}
\bibliography{IEEEfull,sample-bibliography}

\end{document}

\section{Analysis of the Algorithm} \label{sec:analysis}

This section presents an analysis of \packdrop (Algorithm~\ref{alg::packdrop}). 
Symbols presented in this section are available on Tables~\ref{tab:algo:symbols}~and~\ref{tab:analysis:symbols}.
The complexity of the information propagation ($Gossip$) has being evaluated as $O(log_{fout}n)$~\cite{grapevine},
where $fout$ is the $Gossip$ fanout and $n$ is the number of PEs in the system.
Here we use $fout=2$, in order to avoid network congestion. 

For the sake of simplicity, in the remainder of this analysis, the number of tasks in the system
will be referred to as $m$, and the costs for computation and communication will be represented as $p_c$ and $c_c$, respectively.
We also assume $c_c > p_c$ for all concurrent scenarios, since communication costs are several orders of magnitude higher than computational costs.
$C(A)$ is referred here as the complexity class of a given workload $A$, similar to its total cost.
Unmentioned lines are assumed to have non-varying costs, and thus will not interfere in the asymptotic analysis.

Lines~2~and~3 are global reductions, which have a well-known cost of $O(log\ n)$.
Lines~8~and~11 are concurrent, so their cost will be the maximum among them:
\begin{equation}
  max(C(BatchAssembly),C(Gossip))
\end{equation}

We also know that the worst case for \batchassembly (Algorithm~\ref{alg::packcreation}) is rather unrealistic, 
since it would assume that a single PE contains $m$ tasks and a single task may have a load greater than the average system load, being $O(m-1)$, assuming $1$ would not be migrated, asymptotically, $O(m)$.

\begin{table}[t]
	\caption{List of symbols used in the Analysis of the Algorithm.}
	\centering	
	\begin{tabular}{l l}
		\toprule
		\textbf{Symbol}		& \textbf{Meaning} \\
		\midrule
		$fout$			& $Gossip$ protocol's \textit{fanout} \\ 
		$p_c$			& Computational (processing) base cost \\
		$c_c$			& Communication base cost \\
		$C(A)$			& The complexity class of a given function $A$ \\
		$m$				& Number of tasks in the system \\
		$m_l$			& $max(|T|)$ in an overloaded PE \\
		\bottomrule
	\end{tabular}
	\label{tab:analysis:symbols}
\end{table}

Thus, the cost of lines~8~and~11 is:
\begin{equation}
 max((p_c\times m),\ (c_c\times log\ n))
\end{equation}
and since $c_c$ is several orders of magnitude bigger than $p_c$, we could assume $C(BatchAssembly) \in C(Gossip)$, which makes the 
complexity of these lines to $O(log\ n)$.

Finally, line~14 will have a complexity equal to the largest number of packs migrated by an overloaded PE.
Let $ps$ be the average number of tasks inside of an $LT$, and $m_l$ the maximum number of tasks in a given overloaded PE.
At this step, a solution without \textit{Batch Task Migration} would have a cost of $c_c\times m_l$, while our approach will divide this complexity by $ps$. 
This is the most expensive part of Algorithm~\ref{alg::packdrop}, and as such it is the most interesting one to optimize.
Our final asymptotic complexity is:

\begin{equation}
 C(PackDrop) = O(m_l/ps) + O(log\ n)
 \label{eq:worstcase}
\end{equation}

This shows that determining a good $ps$ value is crucial to achieve the best performance with this algorithm.
Higher values of $ps$ will lower communication complexity, but may lead to an imprecise scheduling.
In our implementation, we chose to vary the value of $ps$ around a base value of $2$, according to system load and characteristics, as described in Equation~\ref{eq:ps}.
\batchassembly (Algorithm~\ref{alg::packcreation}) stores tasks in $LT$ in an increasing load order.
So, even though our average pack should have around $2$ tasks, as their load vary, we are able to include more tasks at a lower communication cost.
This way, we attempt precise load balance, while still preserving task locality after migration.

It is important to note that different applications may react differently to different values of $ps$.
This is specially related to application load variance.
The $ps$ factor may be fine-tuned to each specific application and platform, but we believe a reasonable, generic, approach such as the provided by Equation~\ref{eq:ps} is enough to provide balance to most applications and systems, while still harvesting the advantages of the \textit{Batch Task Migration} approach.

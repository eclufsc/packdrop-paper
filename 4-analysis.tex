\section{Analysis of the Algorithm} \label{sec:analysis}

This section presents an analysis of the Parallel Algorithm~\ref{alg::packdrop}~(PackDrop). 
Symbols presented in this section are available on Tables~\ref{tab:algo:symbols}~and~\ref{tab:analysis:symbols}.
The complexity of the information propagation ($Gossip$) has being evaluated as $O(log_{fout}n)$~\cite{grapevine},
where $fout$ is the fanout for the algorithm and $n$ is the number of PEs in the system.
Here we use $fout=2$, in order to avoid network congestion. 

\begin{table}[!ht]
	\caption{List of symbols used in the Analysis of the Algorithm}
	\centering	
	\begin{tabular}{l |l}
		Symbol			& Meaning\\		\hline
		$fout$			& $Gossip$ protocol's \textit{fanout}						 \\ 
		$p_c$			& Computational (processing) base cost				 \\
		$c_c$			& Communication base cost							 \\
		$C(A)$			& The total cost of a given fuction $A$				 \\
		$m$				& Number of tasks in the system						 \\
		$ps$				& The average number of tasks in $LT$s				 \\
		$m_l$			& $max(|T|)$ in an overloaded PE						 \\
	\end{tabular}
	\label{tab:analysis:symbols}
\end{table}

For the sake of simplicity, in the remainder of this analysis, the number of tasks in the system
will be referred to as $m$, and the costs for computation and communication will be represented as $p_c$ and $c_c$, respectively.
We also assume $c_c > p_c$ for all concurrent scenarios, since communication costs are several orders of magnitude higher than computational costs.
$C(A)$ is refered here as the total cost for a given workload $A$.
Unmentioned lines are assumed to have non-varying cost, and thus will not interfere in the asymptotic analysis.

Lines~2~and~3 are global reductions, which have a well known cost of $O(log\ n)$.
Lines~8~and~11 are concurrent, so their cost will be the max among both:
\begin{equation}
  max(C(BatchCreation),C(Gossip))
\end{equation}
We also know that the worst case for $BatchCreation$ (Algorithm~\ref{alg::packcreation}) is rather unrealistic, 
since it would assume that a single PE contains $m$ tasks and a single task may have a load greater than the average system load, being $O(m-1)$, assuming $1$ would not be migrated, asymptotically, $O(m)$.

This takes lines~8~and~11 cost to:
\begin{equation}
 max((p_c\times m),\ (c_c\times log\ n))
\end{equation}
and since $c_c$ is several orders of magnitude bigger than $p_c$, we could assume $C(BatchCreation)\in C(Gossip)$, which makes the 
complexity of these lines to $O(log\ n)$.

Finally, line~14 will have a complexity equal to the largest number of packs migrated by an overloaded PE.
Let $ps$ be the average number of tasks inside of a $LT$, and $m_l$ the maximum number of tasks in a given overloaded PE.
At this step, a solution without Batch Task Migration would have a cost of $c_c\times m_l$, while our approach will divide this complexity by $ps$. 
This is the most expensive in Algorithm~\ref{alg::packdrop}, and as such it is the most interesting one to optimize.
Our final asymptotic complexity will be:
\begin{equation}
 C(PackDrop) = O(m_l/ps) + O(log\ n)
 \label{eq:worstcase}
\end{equation}

This shows that determining a good $ps$ value is crucial to achieve the best performance with this algorithm.
Higher values of $ps$ will lower communication complexity, but may lead to an imprecise scheduling.
In our Implementation, we chose a moderate value of $ps$, of around $2$ average tasks, varying to smaller values according to system load.
This guarantees a precise scheduling, but still gives margin to have several small tasks migrating at once, saving communication costs.

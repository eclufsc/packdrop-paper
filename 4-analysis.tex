\section{Analysis of the Algorithm}

This section presents an analysis of the Parallel Algorithm~\ref{algo:packdrop}~(PackDrop). 
The complexity of the information propagation \textit{Gossip} have being evaluated as $O(log_fn)$ on previous work~\cite{grapevine},
where $f$ is the fanout for the algorithm and $n$ is the number of PEs in the system.

For the sake of simplicity, in the remainder of this analysis, the number of tasks in the system
will be refered to as $m$, and the costs for computation and communication will be represented as $p_c$ and $c_c$, respectively.
We also assume $c_c > p_c$ for all applying scenarios.
Unmentioned lines are assumed to have nonvarying cost, and thus will not interfer in the asymptotic analysis.

Lines~2~and~3 are reductions, which have a well known cost of $O(lg\ n)$ (equal to $O(log_2n)$).
Lines 8 and 11 are concurrent, so their cost will be the max among both:
\begin{equation*}
  max(T(PackCreation),T(Gossip))
\end{equation*}
We also know that the worst case for $PackCreation$ (Algorithm~\ref{alg::packcreation}) is rather unrealistic, 
since it would assume that a single PE contains $m$ tasks and a single task can have $load(task) > \overline{load}$, being $O(m-1)$, assuming $1$ would not be migrated, asymptotically, $O(m)$.

%Aside from this case, we could assume, for fixed $m$ applications and locality based overdecomposition, a complexity $O(m/n -t)$, 
%where $t$ are the tasks that stay on the PE so it remains balanced.
%The value of $t$ will vary with the imbalance of the entire system, the worst case would be $t=1$, so we assume $T(Pack Creation)\in O(m/n)$.

This takes lines 8 and 11 cost to:
\begin{equation*}
 max(O(p_cm),O(c_clog_fn))
\end{equation*}
and since $c_c$ is several orders of magnitude bigger than $p_c$, we could assume $T(PackCreation)\in T(Gossip)$, which makes the 
complexity of these lines to $O(log_fn)$.

Finally, line 14 will have its cost associated directly with the cost of Algorithm~\ref{alg::packcreation}, since it will account $c_c$ for every pack that was created then.
Let $ps$ be the mean number of tasks inside of a pack, and $m_l$ the maximum number of tasks in a given overloaded PE.
%This is where our approach presents gains in lessening communication costs.
At this step, a solution without Task Packing would have a $c_c$ associated with $O(m_l)$, of value $O(log_nm)$ in the average case.
This happens because the higher the $m/n$ ratio, the more tasks will be migrated in a decision process similar to the one made on Algorithm~\ref{alg::packcreation}, but as overdecomposition increases and imbalance decreases, $m_l$ is unable to grow as much, following a logarithmic curve. 

In our proposed solution, the complexity of line~14 will be divided by $ps$, enabling even higher scalability. Having a fanout $f$ of $2$, we have an asymptotic worst case of:
\begin{equation}
 T(PackDrop) = O(m/ps) + O(lg\ n)
 \label{eq:worstcase}
\end{equation}
and an average case of:
\begin{equation}
 T(PackDrop) = O(log_n(m/ps))+O(lg\ n) = O(lg\ n)
\end{equation}
\section{Related Work} \label{sec:rw}

%\todo[inline]{Organizar o raciocínio sendo empregado para listar/citar esses trabalhos todos. Por exemplo, falamos de work stealing sem explicar porque não se aplica perfeitamente para o nosso caso nem porque não fazemos o empacotamento para work stealing.}
%
%\begin{itemize}
%	\item Challenges in global scheduling
%	\item Scalable approaches: Hierarchical, Distributed
%	\item Hierarchical: Bottleneck limitations
%	\item Distributed: Diffusive, iterative, long convergence time
%	\item Distributed: Refine-oriented, excessive communication
%	\item Distributed: Work stealing, non-applicable for iterative applications
%	\item Bin Packing: BinLPT and iteration affinity in loop scheduling
%	\item Refine-oriented bin packing to solve aforementioned issues.
%\end{itemize}

Global rescheduling is a well-studied problem in High Performance Computing (HPC)~\cite{Deveci2015,Zheng2010}.
Redistributing the workload among PEs of a parallel system is a way to mitigate load imbalance created by dynamic applications.
This is done in order to achieve strong scalability, and thus, efficient use of computing resources.
In this section, we will discuss how different approaches seek to perform load balancing in distributed systems, why they lack scalability, and how we intend to mitigate migration, communication, and scalability issues in our proposed solution.

%We define efficient global rescheduling as the ability of redistributing work between PEs during runtime while reducing the total execution time of the application.
%The more this time is diminished, the more efficient a technique is.
%Specifically, in the iterative application domain, this rescheduling is performed between iterations, while the application is ``paused".
%Thus, the rescheduler must perform its work fast, in order to provide benefits to the application.

In the centralized domain, strategies implement a variety of heuristics in order to achieve an homogeneous distribution of load.
Although centralized algorithms are used the most, their sequential and data dependent approach lacks scalability and new approaches must be pursued as systems grow.
Different approaches have been used to scale scheduling strategies, being \textit{hierarchical} and \textit{distributed} the most widespread ones.


Hierarchical algorithms will work differently in different granularity levels, exploring parallelism and delivering better performance~\cite{hybrid,nuco}.
These strategies are able to acquire as much static system information as the centralized techniques, while taking advantage of system parallelism.
Some hierarchical strategies have used topology-aware approaches to increase mapping affinity~\cite{nuco,hwtopo}, while others rely on a hypergraph representation to precisely describe application communication patterns~\cite{ZoltanParHypRepart07}.
However, these approaches still tend to create communication bottlenecks and may take too long to aggregate the necessary information to perform the rescheduling itself.
As parallel systems grow, the cost of having all this system information increases, and tend become inefficient.

\textit{Work stealing} is one of the most broadly used distributed techniques for balancing load in parallel systems~\cite{DBLP:journals/ijpp/YangH18,Janjic2013}.
The essence of work stealing makes it a very effective solution for highly irregular parallel and distributed applications.
However, work stealing may not be as effective, since its concurrent and randomized nature may interfere with iterative application execution~\cite{lifflander2012work}.

Also in the distributed domain, diffusive techniques have been used to irradiate work in an iterative fashion among PEs~\cite{diffus}.
Although such an approach is interesting, since it may not impact much communication costs, it may also have a high convergence time, easily becoming inefficient in very imbalanced scenarios.
Refinement-based distributed techniques, on the other hand, are able to provide fast and efficient rescheduling decisions without knowing too much about system information~\cite{grapevine}.
The main disadvantage of these techniques is the lack of affinity in migrated tasks, diminishing task locality, and thus, increasing their total communication load.

In the loop scheduling domain, a \textit{Bin Packing} oriented approach has been able to exploit iteration affinity by adaptively partitioning loops~\cite{Castro-Penna-WSCAD:2017}.
Due to its greedy approach, this strategy can efficiently distribute work among chunks before scheduling, increasing the overall application performance.

In this work we propose a new approach for the distributed rescheduling domain.
We adopt an approach similar to \textit{Bin Packing} to preserve affinity and diminish communication overheads in our novel distributed rescheduler.
Thus, our completely decentralized approach intends to take profit from both distributed and affinity oriented strategies.
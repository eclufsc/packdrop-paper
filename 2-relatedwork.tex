\section{Related Work} \label{sec:rw}

\todo[inline]{Organizar o raciocínio sendo empregado para listar/citar esses trabalhos todos. Por exemplo, falamos de work stealing sem explicar porque não se aplica perfeitamente para o nosso caso nem porque não fazemos o empacotamento para work stealing.}

Global rescheduling is a well studied theme in high performance computing~\cite{Deveci2015,grapevine,nuco,hybrid,ZoltanParHypRepart07,diffus,Zheng2010}. 
In the centralized domain, strategies implement a variety of heuristics in order to achieve an homogeneous distribution of load.
Although centralized algorithms are used the most, they lack scalability and new approaches must be pursued as systems grow.

Refinement based solutions in centralized load balancing are very effective in dealing with low imbalance or high migration costs~\cite{MenonPHD}. 
Since these strategies use a limited migration heuristic, they are able to efficiently balance load with a low overhead.
However, since refine strategies limit a total number of migrations, they cannot deal with very unbalanced systems.
\todo[inline]{Citar balanceadores centralizados é mesmo necessário? Não parecem faltar referencias de distribuídos para colocar algo distante como relacionado}

Different approaches have been used to scale scheduling strategies.
Hierarchical algorithms will work differently in different levels, exploring parallelism and delivering better performance~\cite{nuco,hybrid,GengbinThesis}.
Strategies in this domain have been able to ensure scalability so far, but are still limited by a centralized beginning and data dependency.

Also in the hierarchical domain, multilevel graph partitioning has been used to efficiently reschedule tasks~\cite{ZoltanParHypRepart07}.
The hypergraph abstraction accurately represents communication, which provides more precision for load balance~\cite{commaware,Bathele2011graph}.
However, the cost of this representation is very high for a first task mapping process, being more recommended for repartitioning.

Work stealing is one of the most broadly used techniques for balancing load in parallel systems~\cite{DBLP:journals/ijpp/YangH18,Janjic2013}.
The completely distributed essence of this techniques make them very effective scheduling solutions for parallel and distributed irregular applications.

Diffusive load balancing is a completely distributed solution that may benefit iterative applications~\cite{diffus}.
These schedulers irradiate work from overloaded PEs to their neighbors, in an effort to achieve load balance in an iterative fashion.
Very unbalanced systems suffer with this kind of approach, since the scheduler may take too long to reach a solution.

\textit{Grapevine} is a completely distributed refinement-based scheduling solution.
It uses probabilistic transfer of load and epidemic communication protocols to achieve scalability~\cite{grapevine}.
We intend to use what was created in \textit{Grapevine} and reduce its communication, using the \textit{Task Packing} approach.

\textit{BinLPT} is a centralized workload-aware loop scheduler that adopts a greedy bin packing heuristic to adaptively partition the iteration space of irregular parallel loops into several chunks so as to amortize load imbalance while minimizing the number of chunks that are produced~\cite{Castro-Penna-WSCAD:2017}. Thus, runtime scheduling overheads can be reduced and iteration affinity may be exploited efficiently. Then, it assigns the heaviest chunks to the least overloaded cores, and then iteratively assigns lighter chunks to more heavily loaded ones. Finally, whenever a core finishes computing all its chunks, it steals a chunk from other one that still has work left.

\todo[inline]{Add related work to MigBSP~\cite{MigPFrighi} \@ Pilla -- MigPF é centralizado, não vejo o que ganhamos em informação nova citando ele. Também não vale a pena correr o risco de ter muitas autocitações.}

\todo[inline]{Add a final paragraph to explain why our approach solve the aforementioned problems.}
